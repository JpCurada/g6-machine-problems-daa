\documentclass{article}
\usepackage{graphicx}
\usepackage{longtable}
\usepackage{url}

\title{Machine Problem \#3}
\author{Florence Lee Cansino}
\date{June 2025}

\begin{document}

\maketitle

\section{IV. Algorithm Comparison}

\begin{longtable}{|p{2.5cm}|p{4.3cm}|p{4.3cm}|p{4.cm}|}
\hline
\textbf{Algorithm} & \textbf{Strengths} & \textbf{Weaknesses} & \textbf{Real-World \newline Applications} \\
\hline
\endfirsthead
\hline
\textbf{Algorithm} & \textbf{Strengths} & \textbf{Weaknesses} & \textbf{Real-World Applications} \\
\hline
\endhead

Insertion Sort &
\begin{itemize}
\itemindent=-13pt
\item Simple and easy to implement (Wandy, 2022)
\item Works well with small and nearly sorted data (Wandy, 2022)
\item Requires minimal space as it is stable and in-place (Wandy, 2022)
\end{itemize}
&
According to Youcademy (n.d.), the disadvantages of Insertion Sort are as follows:
\begin{itemize}
\itemindent=-13pt
\item Less suitable for larger data sets
\item Performs a lot of element shifts
\item Cannot be divided into separate tasks to run at the same time.
\end{itemize}
&
\begin{itemize}
\itemindent=-13pt
\item Online leaderboards in applications with score and ranking system (Broussard, 2024)
\item Mobile phone contact lists (Broussard, 2024)
\item Sorting tasks in task management apps (Broussard, 2024)
\item Song playlist management (Broussard, 2024)
\end{itemize}
\\
\hline

Binary Search &
\begin{itemize}
\itemindent=-13pt
\item Efficient in sorted large data sets (Jaro Education, 2023)
\item Fast lookup with O(log n) time complexity (GeeksforGeeks, 2025)
\item Reduced search time because it eliminates half of the array every step (GeeksforGeeks, 2025)
\end{itemize}
&
Binary Search has several limitations as stated by Jaro Education (2023):
\begin{itemize}
\itemindent=-13pt
\item Only applicable for sorted arrays
\item Complex and unnecessary for small data sets
\item Can be inefficient if implemented recursively with limited memory 
\end{itemize}
&
Youcademy (n.d.) highlights that Binary Search is applied in real-world scenarios such as:
\begin{itemize}
\itemindent=-13pt
\item Filtering items when searching for a product in e-commerce 
\item Used in database indexing
\item Searching for a word in a dictionary or digital word lists
\end{itemize}
\\
\hline

Russian Multiplication Method &
\begin{itemize}
\itemindent=-13pt
\item Only requires doubling and halving when implementing (MathCurious, 2019)
\item Suitable for manual and mental computations (Peterson, 2024)
\item Does not require advanced mathematics (MathCurious, 2019)
\end{itemize}
&

\begin{itemize}
\itemindent=-13pt
\item Limited applications as it cannot be applied to negative numbers (MathCurious, 2019)
\item No direct counterpart for division (Peterson, 2024)
\item Not practical for larger numbers (MathCurious, 2019)
\end{itemize}
&
MathCurious (2019) highlights the following real-world uses of this algorithm:
\begin{itemize}
\itemindent=-13pt
\item Practical for teaching concepts of mathematics
\item Useful as a mental computation technique
\item Teaching concepts of binary number system
\end{itemize}
\\
\hline

Josephus Problem &
According to OpenGenus (n.d.), the Josephus Problem has the following advantages:
\begin{itemize}
\itemindent=-13pt
\item Can be implemented for any number of people and interval sizes
\item Can have iterative, recursive, and mathematical formula approaches
\item Performs greatly because of O(n) time complexity and O(1) space complexity 
\end{itemize}
&
\begin{itemize}
\itemindent=-13pt
\item Might cause stack overflow in large inputs if recursively implemented (Mohandas, A.) 
\item Have limited implementations compared to other algorithms (Mohandas, A.)
\end{itemize}
&
\begin{itemize}
\itemindent=-13pt
\item Round-Robin process scheduling in operation systems (Trivedi, 2021)
\item Used to solve circular linked lists (Trivedi, 2021)
\item Used to solve network topology problems (Singh, 2024)
\end{itemize}
\\
\hline


\end{longtable}

\clearpage
\begin{thebibliography}{}

\bibitem {insertion} Wandy, J. (2022, March 24). \textit{The Advantages and Disadvantages of Sorting Algorithms}. Sciencing. \url{https://www.sciencing.com/the-advantages-disadvantages-of-sorting-algorithms-12749529/}
\bibitem {insertion} Youcademy. (n.d.). \textit{Advantages and Disadvantages of Insertion Sort Algorithm}. Youcademy. \url{https://youcademy.org/advantages-disadvantages-of-insertion-sort/#disadvantages-of-insertion-sort}
\bibitem {insertion} Broussard, A. (2024, January 25). \textit{Insertion Sort: A Fundamental Sorting Algorithm}. LinkedIn. \url{https://www.linkedin.com/pulse/insertion-sort-fundamental-sorting-algorithm-ashley-m-broussard-mvwle/}
\bibitem{binarysearch} Jaro Education. (2023, August 16) \textit{Binary Search Algorithm: Benefits, and Examples}. Jaro Education. \url{https://www.jaroeducation.com/blog/binary-search-algorithm/}
\bibitem{binarysearch} GeeksforGeeks. (2025, May 12) \textit{Binary Search Algorithm - Iterative and Recursive Implementation}. GeeksforGeeks. \url{https://www.geeksforgeeks.org/binary-search/}

\bibitem{russian-multiplication} Peterson, D. (2024, February 2) \textit{Russian Peasant Multiplication: How and Why}. The Math Doctors. \url{https://www.themathdoctors.org/russian-peasant-multiplication-how-and-why/}
\bibitem{russian-multiplication} MathCurious. (2019, December 29) \textit{The Russian Multiplication Method}. MathCurious. \url{https://www.mathsisfun.com/numbers/russian-peasant-multiplication.html}
\bibitem{josephus-problem} Mohandas, A. (n.d.) \textit{Josephus Problem}. OpenGenus. \url{https://iq.opengenus.org/josephus-problem/}
\bibitem{josephus-problem} Trivedi, A. (2021, November 8) \textit{Josephus Circle Using Circular Linked List}. OpenGenus. \url{https://www.prepbytes.com/blog/linked-list/josephus-circle-using-circular-linked-list/}
\bibitem{josephus-problem} Singh, M. (2024, January 13) \textit{Exploring Josephus Interconnection Networks : Unveiling Architecture, Applications, and Recent Advancements}. IJSRCSEIT. \url{https://ijsrcseit.com/home/issue/view/article.php?id=CSEIT2410132}
\end{thebibliography}

\end{document}
